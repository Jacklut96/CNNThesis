\documentclass[../main.tex]{subfiles}
\begin{document}
\label{stateofart}
\thispagestyle{empty}

Amongst the different types of cancer, Lung Cancer stands out as one of the most common and the deadliest of them all, causing almost 1.6 million deaths every year on a global basis.\cite{Wasserman2015} When detected in later stages (III - IV) it's almost always considered unoperable and almost no therapy will prove successfull. Only less than 5\% of people diagnosed with lung cancer survive for more than 10 years, but it has been proved that detecting it at earlier stages (I and II) leads to drastic augments in survaivablity rates,\cite{CancerResearchUK} as it becomes possible to perform possibly effective therapies. Thus comes the need of being able to detect lung cancer as soon as possible, meanwhile differentiating between benign and malignant nodules in order to provide the best possibile treatment to the patient. Imaging techniques as Computer Tomography (CT) and Positron Emition Tomography (PET) are becoming, as of today, more and more of a standard as aiding tools to radiologists in performing diagnosis in patients which are suspected (or are already confirmed) to have lung cancer.\cite{Indicators2017} Time, as stated beforehand, is a critical factor in lung cancer diagnosis and the use of screening techniques like the ones already mentioned has proved to improve the amount of lung nodules and lung cancers detected at early stages (compared to chest radiography).\cite{AlMohammad2017} Despite the usage of CT and PET, though, time still comes as a scarce resource for radiologists, and the possibility of committing mistakes is always present. A nodule might come by undetected by the human eye if too small or well hid in proximity of the rib cage, whilst a bening nodule might be exchanged for a malignant or viceversa. It is then pretty much obivious the necessity of Computer Aided Detection (CAD) in providing new faster and more precise ways to perform diagnosis. Where aided by CAD systems, radiologists show less diagnostic errors and false negatives, also being more accurate than a single or even a double radiologist reading.\cite{AlMohammad2017} The concept of CAD is based on the conversion of medical images in data which can be mined, analyzed and combined with already acquired knowledge in order to provide support for decision making in medicine. This process, known as \textit{Radiomics}, allows for the extraction of features from images and their subsequent analisys, hence allowing for the developing of models with the potential to improve the accuracy of diagnosis and prognosis.\cite{Gillies2016} 
\vspace{5mm}
\section{Radiomics}
Radiomics allows to detect large datasets, containing huge amounts of data, and extract valuable information frome those datas. Informations, in the field of lung cancer detection, refer to features and characteristics of nodules: shape, position, intensity, texture, wavelet, etc. are all features which can be extracted from medical images and analyzed in order to obtain support in decision making. \cite{Chen2017} As of today radiologists manually identify, on every single CT slice (from 256 to more than 400 CT slices per patient), a \textit{Region of Interest (ROI)}, containing the supposed nodule. Such ROI will then be analyzed and on it feature extraction will be performed. This process, though, turns out to be extremely tedious and time consuming for radiologist, providing us with an extremely important problem: to find an automatized way to perform lung nodules detection. 

\end{document}