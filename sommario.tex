\documentclass[../main.tex]{subfiles}
\thispagestyle{empty}
\chapter*{Sommario}

Oggigiorno, i sistemi di autenticazione biometrica sono oggetto di grande interesse, sia da parte della comunità scientifica, sia da parte degli utenti finali, che si trovano ad interfacciarsi con essi con una frequenza sempre maggiore. Tali tecnologie stanno progressivamente sostituendo i mezzi di autenticazione tradizionali, basati su qualcosa che l'utente conosce (come un PIN o una password) o possiede (come un badge elettronico). Il riconoscimento biometrico si pone come un'alternativa estremamente valida a tali sistemi e, risolvendone le principali criticità, garantisce un livello di sicurezza ancora più elevato. In questo panorama, il progetto HaVIRAS si pone l'obbiettivo di creare un prototipo di un sistema di autenticazione integrato, in grado di identificare l'utente basandosi sul pattern vascolare contenuto nel palmo della mano. Tale tratto biometrico risulta unico, stabile, e acquisibile in maniera pratica e non invasiva: per tali ragioni, questa tecnologia vive oggi una rapida diffusione, ritagliandosi una fetta sempre maggiore del mercato del riconoscimento biometrico. Nel dettaglio, il target di questo progetto è la realizzazione di un sistema embedded in grado di acquisire una foto della mano dell'utente nello spettro degli infrarossi, processarla al fine di estrarne la caratteristica biometrica, ed identificare, o meno, l'utente, gestendo il confronto tra il template estratto e quelli di riferimento, contenuti nel database. Inoltre, al fine di garantire un ampio campo di applicabilità, l'utente potrà scegliere di integrare il sistema HaVIRAS all'interno di un'infrastruttura di maggiori dimensioni, utilizzandolo per la sola estrazione della caratteristica e lasciando ad un sistema centrale il compito di gestire il database e il confronto tra i templates. Infine, data la ridotta disponibilità di risorse nell'ambito delle applicazioni embedded, si è scelto di esplorare i possibili vantaggi offerti da un'implementazione in hardware tramite l'utilizzo di piattaforme FPGA, al fine di accelerare i processi computazionalmente più onerosi.

