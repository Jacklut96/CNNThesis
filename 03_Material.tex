\documentclass[../main.tex]{subfiles}
\begin{document}
%\chapter{Msterials and Methods}
\label{capitolo6}
\thispagestyle{empty}

Feature extraction involves extracting a higher level of information from raw pixel values that can capture the distinction among the categories involved. This feature extraction is done in an unsupervised manner wherein the classes of the image have nothing to do with information extracted from pixels. Some of the traditional and widely used features are GIST, HOG, SIFT, LBP etc. After the feature is extracted, a classification module is trained with the images and their associated labels. A few examples of this module are SVM, Logistic Regression, Random Forest, decision trees etc.

The problem with this pipeline is that feature extraction cannot be tweaked according to the classes and images. So if the chosen feature lacks the representation required to distinguish the categories, the accuracy of the classification model suffers a lot, irrespective of the type of classification strategy employed. A common theme among the state of the art following the traditional pipeline has been, to pick multiple feature extractors and club them inventively to get a better feature. But this involves too many heuristics as well as manual labor to tweak parameters according to the domain to reach a decent level of accuracy. By decent I mean, reaching close to human level accuracy. That’s why it took years to build a good computer vision system(like OCR, face verification, image classifiers, object detectors etc), that can work with a wide variety of data encountered during practical application, using traditional computer vision. We once produced better results using ConvNets for a company(a client of my start-up) in 6 weeks, which took them close to a year to achieve using traditional computer vision. 

Another problem with this method is that it is completely different from how we humans learn to recognize things. Just after birth, a child is incapable of perceiving his surroundings, but as he progresses and processes data, he learns to identify things. This is the philosophy behind deep learning, wherein no hard-coded feature extractor is built in. It combines the extraction and classification modules into one integrated system and it learns to extract, by discriminating representations from the images and classify them based on supervised data.

One such system is multilayer perceptrons aka neural networks which are multiple layers of neurons densely connected to each other. A deep vanilla neural network has such a large number of parameters involved that it is impossible to train such a system without overfitting the model due to the lack of a sufficient number of training examples. But with Convolutional Neural Networks(ConvNets), the task of training the whole network from the scratch can be carried out using a large dataset like ImageNet. The reason behind this is, sharing of parameters between the neurons and sparse connections in convolutional layers. It can be seen in this figure 2. In the convolution operation, the neurons in one layer are only locally connected to the input neurons and the set of parameters are shared across the 2-D feature map.
IMMAGINE

a. Accuracy:
If you are building an intelligent machine, it is absolutely critical that it must be as accurate as possible. One fair question to ask here is that ‘accuracy not only depends on the network but also on the amount of data available for training’. Hence, these networks are compared on a standard dataset called ImageNet.

b. Computation:
Most ConvNets have huge memory and computation requirements, especially while training. Hence, this becomes an important concern. Similarly, the size of the final trained model becomes important to consider if you are looking to deploy a model to run locally on mobile. As you can guess, it takes a more computationally intensive network to produce more accuracy. So, there is always a trade-off between accuracy and computation.

Apart from these, there are many other factors like ease of training, the ability of a network to generalize well etc. The networks described below are the most popular ones and are presented in the order that they were published and also had increasingly better accuracy from the earlier ones.

\subsection{AlexNet}
This architecture was one of the first deep networks to push ImageNet Classification accuracy by a significant stride in comparison to traditional methodologies. It is composed of 5 convolutional layers followed by 3 fully connected layers, as depicted in Figure 1.
\begin{figure}[htbp] 
\centering 
\includegraphics[width=0.5\textwidth]{alexnet}
\caption{AlexNet} 
\label{alexnet} 
\end{figure} 

AlexNet, proposed by Alex Krizhevsky, uses ReLu(Rectified Linear Unit) for the non-linear part, instead of a Tanh or Sigmoid function which was the earlier standard for traditional neural networks. ReLu is given by 

f(x) = max(0,x)

The advantage of the ReLu over sigmoid is that it trains much faster than the latter because the derivative of sigmoid becomes very small in the saturating region and therefore the updates to the weights almost vanish(Figure 4). This is called vanishing gradient problem.\\
In the network, ReLu layer is put after each and every convolutional and fully-connected layers(FC).

\begin{figure}[htbp] 
\centering 
\includegraphics[width=0.5\textwidth]{relu}
\caption{ReLu} 
\label{relu} 
\end{figure}
Another problem that this architecture solved was reducing the over-fitting by using a Dropout layer after every FC layer. Dropout layer has a probability,(p), associated with it and is applied at every neuron of the response map separately. It randomly switches off the activation with the probability p, as can be seen in the figure below.  

\begin{figure}[htbp] 
\centering 
\includegraphics[width=0.5\textwidth]{dropout}
\caption{Dropout} 
\label{dropout} 
\end{figure}
The idea behind the dropout is similar to the model ensembles. Due to the dropout layer, different sets of neurons which are switched off, represent a different architecture and all these different architectures are trained in parallel with weight given to each subset and the summation of weights being one. For n neurons attached to DropOut, the number of subset architectures formed is 2^n. So it amounts to prediction being averaged over these ensembles of models. This provides a structured model regularization which helps in avoiding the over-fitting. Another view of DropOut being helpful is that since neurons are randomly chosen, they tend to avoid developing co-adaptations among themselves thereby enabling them to develop meaningful features, independent of others.

\subsection{GoogLeNet/Inception}
The GoogLeNet builds on the idea that most of the activations in a deep network are either unnecessary(value of zero) or redundant because of correlations between them. Therefore the most efficient architecture of a deep network will have a sparse connection between the activations, which implies that all 512 output channels will not have a connection with all the 512 input channels. There are techniques to prune out such connections which would result in a sparse weight/connection. But kernels for sparse matrix multiplication are not optimized in BLAS or CuBlas(CUDA for GPU) packages which render them to be even slower than their dense counterparts.

\begin{figure}[htbp] 
\centering 
\includegraphics[width=0.5\textwidth]{googlenet}
\caption{GoogleNet} 
\label{https://stackoverflow.com/questions/47633393/how-to-calculate-the-number-of-layer-for-googlenet} 
\end{figure}
So GoogLeNet devised a module called inception module that approximates a sparse CNN with a normal dense construction(shown in the figure). Since only a small number of neurons are effective as mentioned earlier, the width/number of the convolutional filters of a particular kernel size is kept small. Also, it uses convolutions of different sizes to capture details at varied scales(5X5, 3X3, 1X1).

Another salient point about the module is that it has a so-called bottleneck layer(1X1 convolutions in the figure). It helps in the massive reduction of the computation requirement as explained below.

Let us take the first inception module of GoogLeNet as an example which has 192 channels as input. It has just 128 filters of 3X3 kernel size and 32 filters of 5X5 size. The order of computation for 5X5 filters is 25X32X192 which can blow up as we go deeper into the network when the width of the network and the number of 5X5 filter further increases. In order to avoid this, the inception module uses 1X1 convolutions before applying larger sized kernels to reduce the dimension of the input channels, before feeding into those convolutions. So in the first inception module, the input to the module is first fed into 1X1 convolutions with just 16 filters before it is fed into 5X5 convolutions. This reduces the computations to 16X192 +  25X32X16. All these changes allow the network to have a large width and depth.

Another change that GoogLeNet made, was to replace the fully-connected layers at the end with a simple global average pooling which averages out the channel values across the 2D feature map, after the last convolutional layer. This drastically reduces the total number of parameters. This can be understood from AlexNet, where FC layers contain approx. 90/100 of parameters. Use of a large network width and depth allows GoogLeNet to remove the FC layers without affecting the accuracy.

For this project, the network that best suits our needs is ...., to which the following changes have been made to improve accuracy and performance.
\cite{cnn}
\end{document}