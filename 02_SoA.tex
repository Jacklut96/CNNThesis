\documentclass[../main.tex]{subfiles}
\begin{document}
\label{stateofart}
\thispagestyle{empty}

Imaging techniques as Computer Tomography (CT) and Positron Emition Tomography (PET) are becoming, as of today, more and more of a standard in performing diagnosis in patients which are suspected (or are already confirmed) to have lung cancer. \cite{Indicators2017} Time, as stated beforehand, is a critical factor in lung cancer diagnosis and the use of screening techniques like the ones already mentioned has proved to improve the amount of lung nodules and lung cancers detected at early stages (compared to chest radiography).\cite{AlMohammad2017} Despite the usage of CT and PET time still comes as a scarce resource for radiologists, and the possibility of committing mistakes is always present. Thus comes the need of Computer Aided Detection (CAD) in providing new faster and more precise ways to perform diagnosis. Where aided by CAD systems, radiologists show less diagnostic errors and false negatives, and are more efficient than a single or even a double radiologist reading. \cite{AlMohammad2017} 

The concept of CAD is based on the conversion of medical images in data which can be mined, analyzed and combined with already acquired knowledge in order to provide support for decision making in medicine. This process, known as \textit{Radiomics}, allows for the extraction of features from images and their subsequent analisys, hence allowing for the developing of models with the potential to improve the accuracy of diagnosis and prognosis. \cite{Gillies2016}

As long as lung cancer is taken into consideration the problem of feature extraction from images is of primary importance. First and foremost comes the need to identify nodules in CT lung sections, then such nodules have to be characterized as benign or malignant. \cite{Hussein2017} TO BE CONTINUED
\end{document}