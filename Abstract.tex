\documentclass[../main.tex]{subfiles}
\thispagestyle{empty}
\chapter*{Abstract}

Lung Cancer is one of the most common and most dangerous kinds of cancer. The most crucial aspect in treating lung cancer is detecting it at earlier stages, when it's still possible to develop a working treatment. This project's aim is to develop a tool which, once given the Computer Tomography \textit{(CT)} images resulting from a patient's examination, is able to firstly discern the ones belonging to the lungs and the ones which do not. Afterwards, taken only the lungs relative CT scans, the tool is able to differentiate which lungs are physiologic and which one are pathologic, thus detecting potential malignant nodules. The aforementioned tool is entirely developed by means of Convolutional Neural Networks, also known as \textit{CNN}, mathematical models which aim to replicate the functionality of the human brain in image recognition. The peculiarity of Neural Networks, highly abstracting, is their ability to \textit{"learn from their mistakes"} and improve in their task as more and more data goes under their control. The whole code is written in python, with the usage of \textit{Tensorflow} and \textit{Keras} API.
%TODO: ADD RESULTS' SYNTHESIS

