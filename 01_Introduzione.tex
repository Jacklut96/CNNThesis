\documentclass[../main.tex]{subfiles}
\begin{document}
\thispagestyle{empty}
This document contains the documentation of the "Convolutional Neural Network (CNN) for Lung Cancer Detection" project. This Chapter introduces the problem which this project aims to provide a solution through CNN and their develop.

\section{Context and problem definition}
The Convolutional Neural Network is ment to be employed in diagnostic to overcome the presence of human errors in the valuation of CT and PET images. Furthermore, it proposes a more exact and effective solution in less time and it reduces the work load of the radiologist.
the lung cancer is a uncontrolled and invasive growth of tissue, it begins with small nodule in the peripheral part of the lung where it is also site of inflammatory processes (bronchitis, pneumonia); if it is not detected in early stage, the nodule increases in volume and invades the near structures (pleura and mediastinum). Malignant pulmonary nodules, in early stage, has net contours and easily indistinguishable from inflammatory outcomes. This volume growth vaniches any termination surgical therapy.
The human evaluation of lung nodules can introduce many errors related to the need to provide an evaluation as quikly as possible because of the large amount of images per patient to analyze, of the diffivulty to detect the small tumoral masses not completely vascularized with a low glucose metabolism and reduced captation of contrast medium.
it has to be highlighted the necessity of an early diagnosis that allows a radical and targeted therapeutic approach before invasion of the  surrounding structures by the tumor with the patient's clinical cure.
\vspace{5mm}

Our project exploits the ability to analyze imagins of the CNN to produce an almost immediate evaluation of the patient exam.

\section{Proposed solution}

\vspace{5mm}
\end{document}