\documentclass[../main.tex]{subfiles}
\thispagestyle{empty}
\chapter*{Abstract}

Cancer is an invasive disease caused by a DNA mutation of a cell that stops performing its specific functions and uncontrollably reproduces. Detecting cancer in early stage is  is very difficult to but it is the only way to ensure an effective and fast cure. \textit{Radiomics} is a medical field that studies algorithms in order to extract information from medical images. This field of study plays an important role in the research for a method to speed up diagnosis. 
The algorithms can recognize malignant nodules in \textit{CT} images, but each body part need a specific algorithm thus CT images have to be manually classified in the different anatomical district.
This project's aim is to develop a tool that can automatically classify the different body regions in order to speed up the diagnostic process. This tool is entirely developed by means of \textit{Convolutional Neural Networks}, also known as \textit{CNN}, mathematical models which purpose is to replicate the functionality of the human brain in image recognition. The peculiarity of Neural Networks, highly abstracting, is their ability to \textit{"learn from their mistakes"} and improve in their task as more and more data goes under their control. The whole code is written in python, with the usage of \textit{TensorFlow} and \textit{Keras} API.

%TODO: ADD RESULTS' SYNTHESIS

