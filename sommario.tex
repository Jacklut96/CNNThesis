\documentclass[../main.tex]{subfiles}
\thispagestyle{empty}
\chapter*{Sommario}

 Il cancro è una malattia con prognosi molto spesso infausta, al giorno d'oggi, gli studi dimostrano come il modo più efficace per curarlo sia la diagnosi precoce. Ma anche questa diagnosi precoce risulta molto difficile da ottenere, poiché i noduli in stadio iniziale sono molto difficili da localizzare e identificare. Nello stadio precoce, si presentano come piccoli raggruppamenti di cellule con un'attività metabolica molto elevata, data la piccola dimensione non sono così semplici da identificare con un esame diagnostico. A questo scopo si utilizzano congiuntamente immagini CT e PET: le prime forniscono l'aspetto morfologico dei tessuti del corpo e le seconde una stima quantitativa dell'attività metabolica degli stessi.
 Successivamente queste immagini biomediche vengono sottoposte alla valutazione di medici specialisti i quali  identificano i noduli tumorali. Questa procedura, però, risulta molto lenta e complessa, non esente da casi di errore umano. 
 Con l'avvento di nuove tecniche e architetture per l'analisi di immagini, questo prcesso viene accorciato in termini di tempo grazie all'utilizzo di specifici algoritmi. La \textit{Radiomica} è la branca della medicina che si occupa di segmentazione ed estrazione di caratteristiche dalle immagini.
 Ogni distretto anatomico ha bisogno di un algoritmo specifico per la localizzazione dei noduli e questo comporta una manuale divisione delle immagini nelle varie parti anatomiche da parte dell'operatore. Il nostro obiettivo quindi è lo sviluppo di un architettura che riesca a distinguere i diversi distretti anatomici per promuovere suddivisione autonoma delle immagini per ogni algoritmo di segmentazione. Questo \textit{tool} sfrutta le Reti Neurali Convoluzionali \textit{(Convolutional Neural Networks [CNN])} che sono modelli matematici che imitano il comportamento del cervello umano. Sono una rete di neuroni artificiali che comunicano tra loro e pesano le informazioni per operare una classificazione. La caratteristica delle \textit{CNN} è l'abilità di imparare acquisendo nuovi elementi ogni volta che ci si sottopongono nuove immagini al fine di rendere più solida l'operazione decisionale. Questo porta ad una velocizzazione del processo di diagnosi, automatizzando la divisione in distretti anatomici per sottoporre le immagini direttamente al giusto algoritmo per la localizzazione di tumori. 

